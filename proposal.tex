\documentclass{scrartcl}

\usepackage[hidelinks]{hyperref}
\usepackage[none]{hyphenat}
\usepackage{setspace}
%\doublespace Plz no, I need to proofread this

\title{Which object-oriented architectures could best support QOS-oriented map streaming in a multiplayer persistent C++ game?}
\subtitle{COMP130 - Computing Architecture}
\date{\today}
\author{1707981}

\begin{document}
	\maketitle
	\pagenumbering{arabic}

Games to lookup: Barotrauma, Doom 3?

\section{Proposal}
Free-to-play multiplayer games enjoy a high popularity in poorer countries such as China, Brazil and Mexico, yet the lower broadband speeds in these areas can pose a particular challenge to games with persistent player-driven worlds. These types of worlds require the server to upload level data to each player periodically, such that they can see the current version of an area as they enter it.

Unfortunately, broadband network speeds remain notoriously unreliable especially in developing countries. In order for games to be available to this wider market, they must utilise ways to reduce bandwidth while still providing enough data for the game to be playable. 

Quality-of-service encompasses the concept of restricting traffic in high bandwidth scenarios such that the data most useful to the player is prioritised. However, to utilise such a system, a game's internal architecture--including class structure, design patterns--must be prepared to function with incomplete data.

This essay explores the application of design patterns to quality-of-service streaming in games. First, the significance and impact of low bandwidth on the player's end will be explored and challenged. Then, previous methods and architectures employed by companies in the past will be identified. This includes traditional single-player system architectures employed by successful games, followed by those employed by multiplayer games. A point of interest is whether there is an existing difference. Finally, this essay will compare the most popular architectures in general and attempt to theoretically deduce
 which may be the most efficient
\nocite{*}
\bibliography{references} 
\bibliographystyle{ieeetr}

\end{document}